\section{Fehlerrechnung}
\label{sec:Fehlerrechnung}
Wenn fehlerhafte Größen in Rechnungen verwendet werden, so muss der neue Fehler mittels der Gaußschen Fehlerfortpflanzung berechnet werden. Der neue Fehler ist dann
\begin{equation}
\label{eqt:fehlerfortpflanzung}
\Delta f = \sqrt{\sum_{i=0}^N \bigl( \frac{\partial f}{\partial x_i} \bigr) ^2 \cdot (\Delta x_i)^2 },
\end{equation}
wobei $f$ die zu errechnende Größe bezeichnte und die $x_i$ die fehlerbehaften Größen, von denen $f$ abhängt.
Wenn Mittelwerte zu bestimmen sind, errechnen sich diese durch
\begin{equation}
\label{eqt:mittelwert}
\overline{x} = \frac {1} {N} \sum_{i=1}^N x_i.
\end{equation}
Der zum Mittelwert gehörige Fehler ist
\begin{equation}
\label{eqt:FehlerMittelwert}
\Delta \overline{x} = \frac{1}{\sqrt{N \cdot (N-1)}} \sqrt{ \sum_{i=1}^N (x_i - \overline{x})^2}.
\end{equation}

Lineare Ausgleichsgeraden berechnen sich über
\begin{equation}
\label{eqt:Gerade}
y = a \cdot x + b.
\end{equation}
Die dazugehörigen Parameter $a$ und $b$ sind mittels
\begin{equation}
\label{eqt:a}
a = \frac {\sum_{i=1}^N (x_i - \overline{x}) (y_i - \overline{y})}{\sum_{i=1}^N (x_i - \overline{x})^2}
\end{equation}
und
\begin{equation}
\label{eqt:b}
b = \overline{y} - a \cdot \overline(x)
\end{equation}
zu berechnen.