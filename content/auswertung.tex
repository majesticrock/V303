\newpage
\section{Auswertung}
\subsection{Messung von verrauschten und unverrauschten Signalen}
\label{sec:Auswertung}

Zunächst wird die konstate Spannung des Funktionsgenerators gemessen. Diese wird von dem Oscillator-Ausgang bereitgestellt und beträgt $U_\text{const} = 3.2$V\!. Der Reference-Ausgang liefert eine variable Spannungsamplitude $U_\text{ref}$.
Die Frequenzen sind jeweils einstellbar.

Die Phasenverschiebung der beiden Signale lässt sich mithilfe von Python 3.7.0 auf circa 45° bestimmen. Mit dieser Information und \autoref{eqn:u_out} lässt sich nun auch $U_\text{out}$ berechnen.
Diese beiden sind in den foglenden beiden Messwerttabellen gegeneinander aufgeführt.

\begin{table}[!htp]
	\centering
	\label{tab:Phasenverschiebung}
	\caption{Messdaten der Phasenverschiebung.}
	\begin{subtable}{0.48\textwidth}
		\centering
		\begin{tabular}{S[table-format=4.0] S[table-format=3.1] S[table-format=2.1]}
			\toprule
			{$\Delta\phi$} & {$U_\text{out} / V$} & {$U_\text{out}$} \\
			\midrule
			  0° &  17.5 &  1.3 \\
			 15° &  20.0 &  1.7 \\
			 30° &  22.5 &  1.9 \\
			 45° &  25.0 &  2.0 \\
			 60° &  22.5 &  1.9 \\
			 75° &  20.0 &  1.8 \\
			 90° &  17.5 &  1.5 \\
			105° &  15.0 &  1.0 \\
			120° &  10.0 &  0.5 \\
			135° &   0.0 &  0.0 \\
			150° &  -7.5 & -0.5 \\
			165° & -15.5 & -1.0 \\
			180° & -17.5 & -1.5 \\
			\bottomrule
		\end{tabular}
		\caption{Messwerte ohne Rauschen}
	\end{subtable}
	\begin{subtable}{0.48\textwidth}
		\centering
		\begin{tabular}{S[table-format=4.0] S[table-format=3.1] S[table-format=2.1]}
			\toprule
			{$\Delta\phi$} & {$U / V$} & {$U_\text{out}$} \\
			\midrule
			  0° &  20.0 &  1.3 \\
			 15° &  25.0 &  1.7 \\
			 30° &  27.5 &  1.9 \\
			 45° &  30.0 &  2.0 \\
			 60° &  27.5 &  1.9 \\
			 75° &  25.0 &  1.8 \\
			 90° &  20.0 &  1.5 \\
			105° &  17.5 &  1.0 \\
			120° &  10.0 &  0.5 \\
			135° &   0.0 &  0.0 \\
			150° &  -7.5 & -0.5 \\
			165° & -17.5 & -1.0 \\
			180° & -20.0 & -1.5 \\
			\bottomrule
		\end{tabular}
		\subcaption{Messwerte mit Rauschen}
	\end{subtable}
\end{table}

$\Delta\phi$ beschreibt hier bei die am Phasenverschieber eingestellte Phasenverschiebung zwischen dem Oscillator- und dem Reference-Signal und $U_\text{low}$ beschreibt die am
Low-Pass-Filter gemessene Spannung.

\begin{figure}
  \centering
  \begin{subfigure}{0.32\textwidth}
    \centering
    \includegraphics[width=0.95\textwidth]{content/0deg.png}
    \caption{$\Delta\phi = 0°$}
    \label{fig:0-deg}
  \end{subfigure}
  \begin{subfigure}{0.32\textwidth}
    \centering
    \includegraphics[width=0.95\textwidth]{content/45deg.png}
    \caption{$\Delta\phi = 45°$}
    \label{fig:45-deg}
  \end{subfigure}
  \begin{subfigure}{0.32\textwidth}
    \centering
    \includegraphics[width=0.95\textwidth]{content/90deg.png}
    \caption{$\Delta\phi = 90°$}
    \label{fig:90-deg}
  \end{subfigure}

  \begin{subfigure}{0.32\textwidth}
    \centering
    \includegraphics[width=0.95\textwidth]{content/135deg.png}
    \caption{$\Delta\phi = 135°$}
    \label{fig:135-deg}
  \end{subfigure}
  \begin{subfigure}{0.32\textwidth}
    \centering
    \includegraphics[width=0.95\textwidth]{content/180deg.png}
    \caption{$\Delta\phi = 180°$}
    \label{fig:180-deg}
  \end{subfigure}
  \begin{subfigure}{0.32\textwidth}
    \centering
    \includegraphics[width=0.95\textwidth]{content/270deg.png}
    \caption{$\Delta\phi = 270°$}
    \label{fig:270-deg}
  \end{subfigure}
  \caption{Graphen am Oszilloskop}
  \label{fig: graphen}
\end{figure}

In \autoref{fig: graphen} sind die durch den Tiefpass entstehenden Wellen bei der jeweils angegebenen Phasenverschiebung durch den Phasenverschieber. Diese Graphen entstehen durch das nicht verrauschte Signal.
Für das verrauschte Signal haben die Graphen jedoch dieselbe Form, mit der Ausnahme, dass diese an der $x$-Achse gespiegelt sind.

Die Messdaten aus \autoref{tab: Phasenverschiebung} werden zur besseren Einsicht geplottet und es wird ein cosinusförmiger Fit über die Messwerte gelegt.

\begin{figure}
  \centering
  \includegraphics{plot_phase.pdf}
  \caption{Plot und Fit der Daten der Messung ohne Rauschen.}
  \label{fig:plot_phase}
\end{figure}

\begin{figure}
  \centering
  \includegraphics{plot_phase_rauschen.pdf}
  \caption{Plot und Fit der Daten der Messung mit Rauschen.}
  \label{fig:plot_phase_rauschen}
\end{figure}

Die Graphen werden nach der Form $f(x)=a\cdot \cos(x+b)$ geplottet. Mithilfe von Python 3.7.0 lassen sich die unverrauschten Werte als
\\ \\
\centerline{$a= (24.6 \pm 0.7)$V}
\centerline{$b= -0.80 \pm 0.03$}
\\ \\
bestimmen. Für die verrauschten Werte ergibt sich:
\\ \\
\centerline{$a= (29.5 \pm 0.7)$V}
\centerline{$b= -0.79 \pm 0.02$}
\\ \\
Die Werte für $b$ sind dabei in Radiant-Werten angegben. 

\newpage
\subsection{Messung von Lichtintensität einer LED und einer Photodiode}

Um diesen Teil korrekt auswerten zu können werden einige Einstellungen am Gerät benötigt:

\begin{table}[!htp]
  \centering
  \begin{tabular}{cccc}
    \toprule
     & \multicolumn{3}{c}{Vertärkung (Amplifier)} \\
    Frequenz / Hz & Pre-Amplifier & Lock-In & Tiefpass \\
    \midrule
    300 & 100 & 5 & 2 \\
    \bottomrule
  \end{tabular}
\end{table}

Im Folgenden ist die Gesamtverstärkung von 1000 bereits herausgerechnet. Es kann keine Distanz $r_\text{max}$ bestimmt werden, ab der die Photodiode keine Spannung mehr emittiert. Daher wird die Messung bis zum durch den Messchieber begrenzten Abstand $r_\text{max}$ von $1.3$m durchgeführt.

Die Abstrahlung von einer Lichtquelle wie einer LED ist kugelförmig; daraus lässt sich die abgestrahlte Leistung als
\\ \\
\centerline{$P = I\cdot 4\pi r^2$}
\\ \\
bestimmen. So lässt sich leicht zeigen, dass für die Intensität $I$
\\ \\
\centerline{$I\sim \frac{1}{r^2}$}
\\ \\
gilt. Da die gemessene Spannung linear von der Intensität abhängt, gilt außerdem
\\ \\
\centerline{$U\sim \frac{1}{r^2}$.}
\\ \\
\begin{table}[!htp]
  \centering
  \begin{tabular}{S[table-format=1.1] S[table-format=1.4]}
    \toprule
    {$r$ / m} & {$U_\text{LED}$ / mV} \\
    \midrule
    0.1 & 6.560 \\
    0.2 & 2.640 \\
    0.3 & 1.180 \\
    0.4 & 0.700 \\
    0.5 & 0.500 \\
    0.6 & 0.230 \\
    0.7 & 0.192 \\
    0.8 & 0.156 \\
    0.9 & 0.140 \\
    1.0 & 0.130 \\
    1.1 & 0.116 \\
    1.2 & 0.059 \\
    1.3 & 0.052 \\
    \bottomrule
  \end{tabular}
\end{table}

Die gemessenen Werte sind in \autoref{tab:intensity} zu finden.

Die Spannung $U$ wird in \autoref{fig:plot_intensity} in Abhängigkeit vom Abstand $r$ dargestellt. Dabei besitzen die Skalen auf beiden Achsen logarithmische Werte.
Mithilfe von Python 3.7.0 wird nun eine lineare Regression durchgeführt, sodass sich eine Funktion der Form $\symup{ln}(U) = \symup{ln}(ar + b)$ ergibt.
Dadruch ergeben sich folgende Werte:
\\ \\
\centerline{$a = (-1.89 \pm 0.08)$ ln($\frac{\symup{U}}{\symup{m}}$)}
\\ \\
\centerline{$b = (-2.23 \pm 0.07)$ ln(U)}
\\ \\ 
\begin{figure}
  \centering
  \includegraphics{plot_intensity.pdf}
  \caption{Plot und Fit der empfangen Intensität der Leuchtdiode.}
  \label{fig:plot_intensity}
\end{figure}