\section{Diskussion}
\label{sec:Diskussion}
Im Auswertungsteil wurde bereits auf eine Abweichung der Messwerte für $U_\text{low}$ zu $U_\text{out}$ hingewiesen.
Des weiteren war es schwierig den in der Anleitung geforderten Wert von 10mV für $U_\text{ref}$ einzustellen, daher wird dieser Wert mit einem relativ großen Fehler behaftet sein.

Bedingt durch die Größe des Phasenverschiebers am Gerät, war eine exakte Einstellung der Winkel nicht möglich, weshalb die Werte für $\Delta\phi$ nicht genau sein können.
Das beudet des Weiteren, dass die jeweiligen $\Delta\phi$ in der Messung mit verrauschten Signal ein wenig anders sein können, als in der Messung mit unverrauschtem Signal.
Dennoch sind die Kurven beider Signale sehr ähnlich, weshalb dieser Fehler als eher klein eingeordnet werden kann.

Eine vermutlich größere Fehlerquelle ist die analoge Skala am Low-Pass-Filter an dem die Werte für $U_\text{low}$ abgelesen wurden. Die Skala ist einerseits sehr klein, was ein Ablesen mit bloßem Auge deutlich erschwert,
andererseits gibt es keine Möglichkeit die Skala einzustellen beziehungsweise zu verändern, sodass für unterschiedliche Größenordnungen dieselbe Skala verwendet werden muss. Besagter Fehler wurde in den Rechnung nicht näher betrachtet.

Die Messwerte, die mithilfe des Oszilloskop gemessen wurden, sind vergleichsweise als eher genau einzustufen, da sich diese gut einstellen ließ und mit der Cursor-Funktion konnte eine Amplitude gut abgelesen werden.
Die Abbildungen auf dem Bildschirm des Oszilloskops kamen schnell zum stehen und flackerten wenig bis gar nicht.
\\ \\
Im zweiten Versuchsteil fällt wie bereits in der Auswertung erwähnt auf, dass es nicht möglich war ein $r_\text{max}$ zu bestimmen, ab dem kein Signal mehr an der Photodiode zu erkennen war.
Jedoch gibt es von 1,1m Abstand auf 1,2m Abstand einen vergleichsweiße großen Sprung in der Spannung, der nicht in das restliche Schema passt.
FÜr 1,2m auf 1,3m verhällt sich dies allerdings weiterhin erwartungsgemäß. Eine Weitere Untersuchung auf eine größere Distanz war bedingt durch die Länge der Schiebeleiste nicht möglich.
Eventuell war das Licht der LED ab dieser Stelle so schwach, dass es allein nicht ausgereicht hätte, damit die Photodiode ein Signal ausstößt, allerdings zusammen mit anderem Licht in dem Raum könnte die Intensität groß genug gewesen sein, dass dennoch ein Signal gesendet wurde.
Ein Überpürfen desse war nicht möglich, da sich in dem Raum auch andere Menschen aufgehalten haben, die Licht benötigten.
Allerdings wäre dadurch der Werte von $c$ in der Ausgleichsrechnung von dieser Messung erklärt.